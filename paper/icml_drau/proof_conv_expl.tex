%! TEX root = icml_drau.tex
% Let $\deltat_1 > 0$. 

Consider $(s_{\deltat^2})$ the random trajectory of a near-continuous MDP with time-discretization $\deltat^2$ obtained by taking at each step $k$ an action $a_k$ along $a_k \sim \pi(a| s_{\deltat^2}^k)$ independantly. We have:

\begin{align}
  s_{\deltat^2}^{\lfloor 1/\deltat\rfloor} &= s_{\deltat^2}^0 + \sum_{k=1}^{\lfloor 1/\deltat\rfloor} s_{\deltat^2}^{k} -  s_{\deltat^2}^{k-1} + \bigO(\deltat^2) \\
                    &= s_{\deltat^2}^0 + \sum_{k=1}^{\lfloor 1/\deltat \rfloor} F(s_{\deltat^2}^{k-1}, a_{k-1})\deltat^2 + \bigO(\deltat^2)
                    % &= s_t + \sum_{k=1}^{\lfloor \deltat_1/\deltat\rfloor} \left(F(s_{t}, a_k) + \bigO(k\deltat)\right)\deltat + \bigO(\deltat) \\
                    % &= s_t + \sum_{k=1}^{\lfloor \deltat_1/\deltat\rfloor} F(s_{t}, a_k) \deltat + \bigO(\deltat_1^2) + \bigO(\deltat) 
\end{align}

We define $f(s) \deq \E_{a \sim \pi(s)}\left[F(s, a)\right] = \int_{a\in{\cal A}}F(s, a)\pi(s, a)$. Since $\pi$ and $F$ are bounded and $C^1$, we know that $f$ is $C^1$. We have: 

  \begin{align}
    s_{\deltat^2}^{\lfloor 1/\deltat\rfloor}   & = s_{\deltat^2}^0 + \sum_{k=1}^{\lfloor 1/\deltat\rfloor}
                        f(s_{\deltat^2}^{k-1})\deltat^2 + \sum_{k=1}^{\lfloor 1/\deltat\rfloor} (F(s_{\deltat^2}^{k-1}, a_{k-1}) - f(s_{\deltat^2}^{k-1}))\deltat^2 + \bigO(\deltat^2) \\
     s_{\deltat^2}^{\lfloor 1/\deltat\rfloor}  &= s_{\deltat^2}^0 + \sum_{k=1}^{\lfloor 1/\deltat\rfloor}
                        f(s_{\deltat^2}^{k-1})\deltat^2 + \xi + \bigO(\deltat^2)
  \end{align}
  with $\xi \deq  \deltat^2 \sum_{k=1}^{\lfloor 1/\deltat\rfloor} \left(F(s_{\deltat^2}^{k-1}, a_{k-1}) - f(s_{\deltat^2}^{k-1})\right)$. By definition, we have $\E[\xi] = 0$. Moreover, by using the independance of actions and the boundness of F, there is $\sigma >0$ such that:
  \begin{align}
    \E[\|\xi\|^2] \leq \sigma^2\deltat^3
  \end{align}

  We know that $f$ is $C^1$ on a compact space. Therefore, there is $L_f$ such that $f$ is $L_f$ Lipschitz, and we have:
  \begin{align}
    \|\left(\sum_{k=1}^{\lfloor 1/\deltat\rfloor} f(s_{\deltat^2}^{k-1})\deltat\right) - f(s_{\deltat^2}^0) \| \leq \deltat L_f\sum_{k=1}^{\lfloor 1/\deltat\rfloor} \|s_{\deltat^2}^{k-1} - s_{\deltat^2}^0\|
  \end{align}
  Since $F$ is bounded, we know that $\|s_{\deltat^2}^{k} - s_{\deltat^2}^{k-1}\| \leq C\deltat$. Therefore:
  \begin{align}
    \|\left(\sum_{k=1}^{\lfloor 1/\deltat\rfloor} f(s_{\deltat^2}^{k-1})\deltat\right) - f(s_{\deltat^2}^0) \| &\leq \deltat L_fC\sum_{k=1}^{\lfloor 1/\deltat\rfloor} k\deltat \\
    &= \bigO(\deltat^2) 
  \end{align}
  Therefore: 
  \begin{align}
  s_{\deltat^2}^{\lfloor 1/\deltat \rfloor} =  s_{\deltat^2}^0 + f(s_{\deltat^2}^0) \deltat + \xi + \bigO(\deltat^2) 
  \end{align}

  Therefore, we have $a>0$ such that $\|s_{\deltat^2}^{\lfloor 1/\deltat \rfloor} -  s_{\deltat^2}^0 - f(s_{\deltat^2}^0) \deltat\| \leq \|\xi\| + a\deltat^2$
  
We define $(\tilde s_\deltat)$ the deterministic near-continuous process with time discretization $\deltat$ defined by $\tilde s_\deltat^{k+1} \deq s_\deltat^k + f(s_\deltat^k)\deltat$. We have:

\begin{align}
  \|s_{\deltat^2}^{(k+1)\lfloor 1/\deltat\rfloor}  - \tilde{s}_\deltat^{k+1}\| &\leq \|s_{\deltat^2}^{(k+1)\lfloor 1/\deltat\rfloor} - s_{\deltat^2}^{k\lfloor 1/\deltat\rfloor} - f(s_{\deltat^2}^{k\lfloor 1/\deltat\rfloor})\deltat\| + \|s_{\deltat^2}^{k\lfloor 1/\deltat\rfloor} + f(s_{\deltat^2}^{k\lfloor 1/\deltat\rfloor})\deltat - \tilde{s}_\deltat^{k+1}\|
\end{align}
We know that $\|s_{\deltat^2}^{(k+1)\lfloor 1/\deltat\rfloor} - s_{\deltat^2}^{k\lfloor 1/\deltat\rfloor} - f(s_{\deltat^2}^{k\lfloor 1/\deltat\rfloor})\deltat\| \leq \|\xi_k\| + a\deltat^2$. Moreover:

\begin{align}
  \|s_{\deltat^2}^{k\lfloor 1/\deltat\rfloor} + f(s_{\deltat^2}^{k\lfloor 1/\deltat\rfloor})\deltat - \tilde{s}_\deltat^{k+1}\| &\leq \|s_{\deltat^2}^{k\lfloor 1/\deltat\rfloor} - \tilde{s}_\deltat^{k} \| + \deltat\|f(s_{\deltat^2}^{k\lfloor 1/\deltat\rfloor}) - f(\tilde{s}_\deltat^{k})\| \\
  &\leq (1+L_f\deltat)\|s_{\deltat^2}^{k\lfloor 1/\deltat\rfloor} - \tilde{s}_\deltat^{k} \| 
\end{align}
Therefore, we have:
\begin{align}
 \|s_{\deltat^2}^{(k+1)\lfloor 1/\deltat\rfloor}  - \tilde{s}_\deltat^{k+1}\| &\leq \|\xi_k\| + a\deltat^2 + (1+L_f\deltat)\|s_{\deltat^2}^{k\lfloor 1/\deltat\rfloor}  - \tilde{s}_\deltat^{k}\|  
\end{align}

By induction, and by taking $k = \lfloor t/\deltat\rfloor$:
\begin{align}
  \|s_{\deltat^2}^{k\lfloor 1/\deltat\rfloor}  - \tilde{s}_\deltat^{k}\| \leq \frac{a\deltat}{L_f}\exp(L_ft) + \sum_{j=0}^{\lfloor t/\deltat \rfloor}(1+\deltat L_f)^j\|\xi_j\|  
\end{align}

Therefore, if $\varepsilon >0$, we have :
\begin{align}
  \BP\left(\|s_{\deltat^2}^{k\lfloor 1/\deltat\rfloor}  - \tilde{s}_\deltat^{k}\| > \varepsilon \right) &\leq \BP\left(\sum_{j=0}^{\lfloor t/\deltat \rfloor}(1+\deltat L_f)^j\|\xi_j\| > \varepsilon - \frac{a\deltat}{L_f}\exp(L_ft)\right) \\
                                                                                            &\leq \frac{\E\left[\sum_{j=0}^{\lfloor t/\deltat \rfloor}(1+\deltat L_f)^j\|\xi_j\|\right]}{\varepsilon - \frac{a\deltat}{L_f}\exp(L_ft)}\\
                                                                                            &\leq \frac{\E\left[\|\xi\|\right]}{\varepsilon - \frac{a\deltat}{L_f}\exp(L_ft)}\frac{\exp(L_ft)}{L_f\deltat}\\
  % &\leq \frac{\sqrt{\E\left[\|\xi\|^2\right]}}{\varepsilon - \frac{a\deltat}{L_f}\exp(L_ft)}\frac{\exp(L_f\deltat)}{L_f\deltat}\\
\end{align}
But $\E\left[\|\xi\|\right] \leq \sqrt{\E\left[\|\xi\|^2\right]} \leq \sigma \deltat^{3/2}$. Therefore, we have:
\begin{align}
  \BP\left(\|s_{\deltat^2}^{\lfloor t/\deltat\rfloor\lfloor 1/\deltat\rfloor}  - \tilde{s}_\deltat^{k}\| > \varepsilon \right) & = \bigO(\sqrt \deltat)
\end{align}

Therefore, the  process $t \mapsto s_{\deltat^2}^{\lfloor t/\deltat\rfloor\lfloor 1/\deltat\rfloor}$ converges in probability to $\tilde s$. Furthermore, by a similar argument than in Lemma 1, we know that the discretized process $\tilde s$ converge to the continuous process defined by $\frac{ds}{dt} = f(s_t)$. We can conclude ou result.

% We define for all $0 \leq j \leq \lfloor t/\deltat \rfloor$: $\alpha_j = \omega (1+\deltat L_f)^j$, such that $\sum_j \alpha_j = 1$. Then, we have:
% \begin{align}
%   \BP\left( \sum_{j=0}^{\lfloor t/\deltat \rfloor}(1+\deltat L_f)^j\|\xi_j\| > \varepsilon'\right) & \leq \BP\left( \exists j,  (1+\deltat L_f)^j\|\xi_j\| > \alpha_j\varepsilon'\right) \\
%                                                                                                    & \leq \sum_j \BP\left(  (1+\deltat L_f)^j\|\xi_j\| > \alpha_j\varepsilon'\right)     \\
%                                                                                                    & \leq \sum_j \BP\left(\|\xi_j\| > \omega \varepsilon'\right) \\
%   &\leq \frac{t\sigma^2\deltat^2}{\omega^2 \varepsilon'^2}
% \end{align}
% Moreover, we have $\omega \leq \frac{\exp(L_f t)}{L_f \deltat}$. We have:
% \begin{align}
%   \BP\left( \sum_{j=0}^{\lfloor t/\deltat \rfloor}(1+\deltat L_f)^j\|\xi_j\| > \varepsilon'\right) &\leq  t\sigma\deltat\frac{\exp(L_f t)}{L_f}
% \end{align}
% Finally:
% \begin{align}
%   \BP\left(\|s_{t} - \tilde{s}_\deltat^{\lfloor t/\deltat \rfloor} \| > \varepsilon \right) \leq \BP\left(\sum_{j=0}^{\lfloor t/\deltat \rfloor}(1+\deltat L_f)^j\|\xi_j\| > \varepsilon - \frac{a\deltat}{L_f}\exp(L_ft)\right)
% \end{align}
%%% Local Variables:
%%% TeX-master: "icml_drau"
%%% End:

