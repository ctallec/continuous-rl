\documentclass[11pt]{article}

\usepackage[utf8]{inputenc}
\usepackage[english]{babel}

\usepackage{xcolor}
\usepackage{amsfonts}
\usepackage{amsmath}
\usepackage{amssymb}
\usepackage{amsthm}

\newcommand{\deltat}{{\delta t}}
\newcommand{\gauss}{\mathcal{N}}
\newcommand{\E}{\mathbb{E}}
\newcommand{\TODO}[1]{\textcolor{red}{#1}}
\newcommand{\bigO}[1]{O(#1)}
\newcommand{\Tdo}{TD($0$)}
\newcommand{\tdres}{\delta_\text{TD}}

\newcommand{\deq}{\mathrel{\mathop{:}}=}
\newcommand{\eqd}{=\mathrel{\mathop{:}}}


\title{Baseline centering problem}
\begin{document}
\maketitle

Let $\pi$ be any policy (or set of observed transitions) and let
$\rho^\pi$ be the associated distribution on states.

Let us start again from  the equations on $A^*$, $V^*$ and $b$:
\begin{gather}
	A(s, a) = \max\limits_{a'} A(s, a') + 
		(r(s, a) - b)
		 + 
		\gamma \E_{s'\mid a}\left[
			V(s')
		\right] - V(s)
	\\
	\max\limits_{a} A(s, a) = 0\\
	\E_{s\sim \rho^\pi}\left[V(s)\right] = 0
	.
\end{gather}

Let us prove existence and uniqueness of the solution over $(V,b)$ for a
given function $A$. Moreover, let us prove that when $\gamma\to 1$, $b$
tends to the average reward under the policy derived from $A$.

\paragraph{Existence of a solution.}

Let $\pi_A$ be the (or any in case of ties) policy described by the advantage function $A$,
namely, that returns an action $a^\ast$ that maximizes $A(s,a)$.

Applying the equation to $a^\ast$ we find
\begin{equation}
\label{eq:optbellman}
0=r(s,a^\ast)-b+\gamma \E_{s'|a^\ast} V(s')-V(s)
\end{equation}

This is the Bellman equation for the policy $\pi_A$ with rewards $r-b$.
Shifting rewards by $b$ shifts value functions by $b/(1-\gamma)$. So
under ergodicity, we know that the unique solution is
\begin{equation}
V(s)=V^{\pi_A}(s)-\frac{b}{1-\gamma}
\end{equation}
and averaging under the current empirical policy $\pi$ and using
$\E_{\rho^\pi} V=0$ we find
\begin{equation}
b=(1-\gamma)\E_{s\sim \rho^\pi} V^{\pi_A}(s)
\end{equation}

Conversely, the function $V(s)\deq V^{\pi_A}(s)-\frac{b}{1-\gamma}$ with
$b\deq (1-\gamma)\E_{s\sim \rho^\pi} V^{\pi_A}(s)$ satisfies 
both $\E_{\rho^\pi} V=0$ (by construction) and the Bellman equation for
every pair $(s,a^\ast)$, that is, the Bellman equation for $\pi_A$.

TODO Satisfying Bellman for every pair $(s,a)$ instead of just $(s,a^\ast)$
tells that $A$ is indeed the advantage function of $\pi_A$.

Thus, interestingly, the system provides a well-defined solution for $V$
and $b$ even if $A$ is not perfect (namely, the value function associated
with $\pi_A$). This provides some numerical protection.

\paragraph{The value of $b$ when $\gamma\to 1$.}
We know that when $\gamma\to 1$, the value function for $\pi_A$ takes the form
\begin{equation}
\label{eq:reladv}
	V^{\pi_A}(s) = \frac{\E_{(s,a^\ast)\sim\rho^{\pi_A}}r(s,a^\ast) }{1 - \gamma} +
	U(s) + O(1-\gamma)
\end{equation}
where $U(s)$ is the relative advantage function of the policy $\pi^A$.
The function $U$
is bounded when $\gamma\to 1$.

Since $b=(1-\gamma)\E_{\rho^\pi} V^{\pi_A}$ we just find
\begin{equation}
b=\E_{(s,a^\ast)\sim\rho^{\pi_A}}r(s,a^\ast)+O(1-\gamma)
\end{equation}
namely, $b$ is the average reward under policy $\pi^A$.

\end{document}
