\documentclass{article}


\usepackage{mystyle}
\usepackage{mycommands}
\usepackage{lmodern}

\usepackage{todonotes}
%! TEX root = main.tex
\newcommand{\deltat}{\delta\hspace*{-.3mm}t}
\newcommand{\bigO}[1]{O (#1)}
\newcommand{\reward}{\tilde{r}}
\newcommand{\actionspace}{\mathcal{A}}
\newcommand{\statespace}{\mathcal{S}}


\title{Continuous Reinforcement Learning}

\author{L\'eonard Blier, Corentin Tallec}
  

\begin{document}
\maketitle


\begin{abstract}
Abstract
\end{abstract}

\section{Introduction}
\label{sec:introduction}


\section{Definitions and Notations}
\label{sec:definitions}

\begin{definition}[Continuous Markov Reward Process]
  A continuous markov reward process $(X_{t}, R_{t})_{t\in\BR_{+}}$ is a continuous random process defined by 
  \begin{itemize}
  \item $\mathcal{X}$ is the state set 
  \item $T(x)$ the dynamics and $\Sigma (x)$ the noise
  \item $\mathcal{R}(r|x)$ the reward law
  \item $\gamma$ the discount factor
  \end{itemize}
  The law for the process $X_{t}$ is given by
  \begin{equation}
    \label{eq:eqdiffstoch}
    dX_{t} = T(X_{t}) + \Sigma(X_{t})dB_{t}
  \end{equation}
\end{definition}


\begin{definition}[Discretization of the space]
  A discretization of the space $(\mathcal{X}, \BR_{+})$ is a $(X, \delta t)$ where :
  \begin{itemize}
  \item $X$ is a partition of $X$
  \item $\delta t > 0$
  \end{itemize}
  We define the precision $\delta(X)$ of the partition $X$ to be :
  \begin{equation}
    \label{eq:2}
    \delta(X) = \sup_{s\in X}\sup_{x,y \in s}\|x-y\|
  \end{equation}
\end{definition}

\begin{definition}[Convergence of discretizations]
  We say that a sequence $(X_{k}, \delta t_{k})$ of discretizations of $\mathcal{X}$ converges if $(\delta(X_{k}), \delta t_{k}) \rightarrow 0$. 
\end{definition}


\begin{definition}[Discretization of a Markov Reward Process]
  Let $(\mathcal{S}, T, R, \gamma)$ be a continuous MRP, and $(S, \delta t)$ a discretization of $\mathcal{X}$. We define the approximation of the continuous MRP associated to this discretization as the discrete MRP $(\tilde S, \tilde{T}, \tilde{R}, \tilde{\gamma})$ where :
  \begin{itemize}
  \item $\tilde{T}(s,s') = \BP(X_{t+\delta t}\in s' | X_{t}\in s)$. \todo{Formalize that $X_{t}$ uniform in $s$}
  \item $R(s) \sim \int_{t}^{t+\delta t}R_{t'}dt'$\todo{idem}
  \item $\tilde{\gamma} = \gamma^{\delta t}$ 
  \end{itemize}
\end{definition}


\begin{theorem}
  Let $(X_{t}, R_{t})$ be a continuous MRP, and $(X_{k}, \delta t_{k})$ a sequence of approximations which converges to $\mathcal{X}$. Let $(\tilde{X}^{k}_{t}, \tilde{R}^{k}_{t})$ be the associated discretization. Then, for all $T > 0$,
  \begin{equation}
    (X^{k}_{t}, R^{k}_{t})_{0 \leq t \leq T} \rightarrow (X_{t}, R_{t})_{0 \leq t \leq T}
  \end{equation}
\end{theorem}

We define the operators on $\mathcal{M}$ :
\begin{itemize}
\item $U_{S}\mu = \sum_{s \in S}\frac{\mu(s)}{\ell(s)}1_{s}$
\item $P_{S}\mu = \sum_{s \in S}\frac{\mu(s)}{\ell(s)}\delta_{x_{s}}$
\item Family $(T_{t})_{t\in \BR_{+}}$, $T_{t}\mu$ is the law of $X_{t}$ if $X_{0}\sim \mu$.
\item For a given $\delta t$, $\tilde{T}_{t}\mu$ is the law of $\tilde{\tilde X}_{t}$ if $X_{0}\sim \mu$. We have :
  \begin{equation}
    \tilde{T}_{\delta t} = P_{S}T_{\delta t}U_{S}
  \end{equation}
  and
  \begin{equation}
    \tilde{T}_{t} = \tilde{T}_{\delta t}^{\lfloor t/\delta t\rfloor}
  \end{equation}
\end{itemize}
In the following, $W = W_{2}$ is the Wasserstein distance.
\begin{lemma}
  \begin{equation}
    W(U_{S}\mu, \mu) \leq \delta(S)
  \end{equation}
  \begin{equation}
    W(P_{S}\mu, \mu) \leq \delta(S)
  \end{equation}
  \begin{equation}
    U_{S}P_{S} = U_{S}
  \end{equation}
  \begin{equation}
    P_{S}U_{S} = P_{S}
  \end{equation}
\end{lemma}

\begin{lemma}
  Assume that $T$ is Lipschitz and $\mathcal{X}$ is compact, 
  \begin{equation}
    W(T_{t}\mu, T_{t}\nu) \leq (1+Ct)W(\mu, \nu)
  \end{equation}
\end{lemma}

\begin{proof}
  \todo{todo} IDEA : stability of $T$ and $\Sigma$
\end{proof}

\begin{lemma}
   Let $(S_{k}, \delta t_{k})$ be a sequence of approximations which converges to $\mathcal{X}$ such that $\varepsilon_{k} := \frac{\delta(S_{k})}{\delta t_{k}} \rightarrow 0$. Let $(\tilde{X}^{k}_{t}, \tilde{R}^{k}_{t})$ be the associated discretization. Then, for all $t > 0$,
  \begin{equation}
    W(\tilde T^{k}_{t}\mu, T_{t}\nu) \leq W(\mu,\nu)e^{tC} + o_{k\rightarrow \infty}(1)
  \end{equation}
\end{lemma}

\begin{proof}
  Let $(S,\delta t)$ be a discretization.


  \begin{align}
    W(\tilde T_{\delta t}\mu, T_{t}\nu) &= \\
  W(P_{S}(T_{\delta t}U_{S})^{\lfloor t/\delta t\rfloor}\mu , T_{t}\nu) &\leq \\ W(T_{\lfloor t/\delta t\rfloor \delta t}\mu, T_{t}\mu) + \delta(S) + W((T_{\delta t}U_{S})^{\lfloor t/\delta t\rfloor}\mu , T_{\delta t}^{\lfloor t/\delta t\rfloor}\nu)
  \end{align}
  
  But
  \begin{equation}
  W(T_{\lfloor t/\delta t\rfloor \delta t}\mu, T_{t}\mu) = W(T_{\delta t\{t/\delta t\}}\nu , \nu) \leq C_{1}\delta t
\end{equation}
Moreover :
\begin{align}
  W((T_{\delta t}U_{S})^{n}\mu , T_{\delta t}^{n}\nu) &= W(T_{\delta t}\left(U_{S}(T_{\delta t}U_{S})^{n-1}\mu\right) , T_{\delta t}\left(T_{\delta t}^{n-1}\nu\right)) \\
                                                      &\leq (1+C\delta t)  W(U_{S}(T_{\delta t}U_{S})^{n-1}\mu , T_{\delta t}^{n-1}\nu)\\
  & \leq (1+C\delta t)  (\delta(S) + W((T_{\delta t}U_{S})^{n-1}\mu , T_{\delta t}^{n-1}\nu))
\end{align}
Then, by induction :
\begin{align}
  W((T_{\delta t}U_{S})^{n}\mu , T_{\delta t}^{n}\mu) &\leq (1+C\delta t)\delta(S)\frac{(1+C\delta t)^{n} - 1}{C\delta t} + W(\mu, \nu)(1+C\delta t)^{n}
\end{align}
Therefore :
\begin{align}
  W((T_{\delta t}U_{S})^{\lfloor t/\delta t\rfloor}\mu , T_{\delta t}^{\lfloor t/\delta t\rfloor}\mu) &\leq (1+C\delta t)\delta(S)\frac{(1+C\delta t)^{\lfloor t/\delta t\rfloor} - 1}{C\delta t} + W(\mu, \nu)(1+C\delta t)^{\lfloor t/\delta t\rfloor} \\
                                                                                                      &\leq (1+C\delta t)\delta(S)\frac{e^{tC} - 1}{C\delta t} + W(\mu, \nu)e^{tC}\\
                                                                                                        &= W(\mu, \nu)e^{tC} + o(1)
\end{align}
Therefore, 
\begin{align}
  W(P_{S}(T_{\delta t_{k}}U_{S_{k}})^{\lfloor t/\delta t_{k}\rfloor}\mu, T_{t}\mu) & \leq W(\mu, \nu)e^{tC} + o(1) + \delta(S_{k}) + C'\delta t_{k}  \\
  &= W(\mu, \nu)e^{tC} + o(1)
\end{align}
\end{proof}

\begin{lemma}
  Let $0 < t_{1} < ... < t_{l}<T$.
  \begin{equation}
    W((\tilde X_{t_{1}}, ..., \tilde X_{t_{l}}), (X_{t_{1}}, ..., X_{t_{l}})) \leq W(\mu,\nu)(e^{t_{1}C}+...+e^{t_{l}C}) + o(1)
  \end{equation}
\end{lemma}

\begin{proof}
  Let $P$ be the law of $(X_{t_{1}}, ..., X_{t_{l}})$ with $X_{0} \sim \nu$, and $\tilde P$ be the law of $(\tilde X_{t_{1}}, ..., \tilde X_{t_{l}})$ with $X_{0} \sim P_{s}\mu$.
  For all $\mu, \nu$, we use the notation $\pi_{\mu,\nu}$ for a measure achieving the optimal transport distance up to $\epsilon$.

  We define :
  \begin{align}
    \pi((\tilde x_{1},..., \tilde x_{l}), (x_{1}, ..., x_{l})) = \\ = \pi_{\tilde T_{t_{1}}\mu T_{t_{1}\nu}}(\tilde x_{1}, x_{1})\prod_{i=2}^{l}\pi_{\tilde T_{t_{i}-t_{i-1}}\delta_{\tilde x_{i-1}}, T_{t_{i}-t_{i-1}}\delta_{x_{i-1}}}(\tilde x_{i}, x_{i})
  \end{align}
  Then,
  \begin{align}
    & W((\tilde X_{t_{1}}, ..., \tilde X_{t_{l}}), (X_{t_{1}}, ..., X_{t_{l}}))
    \\ & \leq \int_{(\tilde x_{1},..., \tilde x_{l}), (x_{1},..., x_{l})}\sum_{i=1}^{l}\|\tilde x_{i} - x_{i}\|^{2}\pi((\tilde x_{1},..., \tilde x_{l}), (x_{1}, ..., x_{l}))
         \\ &\leq  W((\tilde X_{t_{1}}, ..., \tilde X_{t_{l-1}}), (X_{t_{1}}, ..., X_{t_{l-1}})) + \int_{(\tilde x_{1},..., \tilde x_{l}), (x_{1},..., x_{l})}\|\tilde x_{l} - x_{l}\|^{2}\pi((\tilde x_{1},..., \tilde x_{l}), (x_{1}, ..., x_{l}))
  \end{align}

  But we have :
  \begin{align}
    \int_{(\tilde x_{1},..., \tilde x_{l}), (x_{1},..., x_{l})}\|\tilde x_{l} - x_{l}\|^{2}\pi((\tilde x_{1},..., \tilde x_{l}), (x_{1}, ..., x_{l}))  \\
    \leq \int_{(\tilde x_{1},..., \tilde x_{l-1}), (x_{1},..., x_{l-1})}W(\tilde T_{t_{l}-t_{l-1}}\delta_{\tilde x_{l-1}}, T_{t_{l}-t_{l-1}}\delta_{x_{l-1}})^{2}\pi((\tilde x_{1},..., \tilde x_{l-1}), (x_{1}, ..., x_{l-1}))\\
    \leq \int_{(\tilde x_{1},..., \tilde x_{l-1}), (x_{1},..., x_{l-1})}(W(\delta_{\tilde x_{l-1}}, \delta_{x_{l-1}})^{2}e^{(t_{l}-t_{l-1})C}+\varepsilon)\pi((\tilde x_{1},..., \tilde x_{l-1}), (x_{1}, ..., x_{l-1}))\\
    \leq \int_{(\tilde x_{1},..., \tilde x_{l-1}), (x_{1},..., x_{l-1})}(\|\tilde x_{l-1} - x_{l-1}\|^{2}e^{(t_{l}-t_{l-1})C}+\varepsilon)\pi((\tilde x_{1},..., \tilde x_{l-1}), (x_{1}, ..., x_{l-1}))
  \end{align}
  
  By induction :
  \begin{align}
    & W((\tilde X_{t_{1}}, ..., \tilde X_{t_{l}}), (X_{t_{1}}, ..., X_{t_{l}}))
    \\ &\leq  W((\tilde X_{t_{1}}, ..., \tilde X_{t_{l-1}}), (X_{t_{1}}, ..., X_{t_{l-1}})) + W(\mu,\nu)e^{t_{l}C}
    \\ &\leq W(\mu,\nu)(e^{t_{1}C}+...+e^{t_{l}C}) + o(1)
  \end{align}
\end{proof}



\begin{proof}[Proof of the theorem]
  We use theorem 13.5 from Billingsley \todo{cite}. We have to check two hypothesis :
  \begin{enumerate}
  \item For all $t_{1}<t_{2}<...<t_{l}<T$, $(X_{t_{1}}^{k}, ..., X_{t_{l}}^{k}) \rightarrow (X_{t_{1}}, ..., X_{t_{l}})$ weakly.
  \item For all $r \leq s \leq t$, $\BE\left[|X_{s}-X_{r}|^{2}|X_{t}-X_{s}|^{2}\right] \leq (F(t) - F(r))^{2}$ where F is a continuous non-decreasing function.
  \end{enumerate}
  For the first point, done in Lemma
  
  For the second point \todo{To formalize}:
  \begin{align}
    \BE\left[|X_{s}-X_{r}|^{2}|X_{t}-X_{s}|^{2}\right] &\leq C(r-s)(t-s)\\
    &\leq C\frac{(t-r)^{2}}{4}
  \end{align}
  Then $F(t) = t\sqrt{C}/2$ solves the problem.
\end{proof}


%!TEX root = main.tex

\section{Intuitions}
\label{sec:intuitions}
Reinforcement learning aims at tackling a wide variety of real world problems.
Among them obviously stand physical problems, where an agent interacts with a
physical environment, e.g.\ the real world.

Resistance to a change of time discretization is a desirable property for
algorithms aiming at attacking the reinforcement learning field in physical
setups: such algorithms should only improve as the time discretization is
refined, as refining the time discretization can only increase the amount of 
eploitable information.

However, some reinforcement learning algorithms such as \emph{Q-Learning} are
ill-behaved in the regime of small time discretization. In what follow, we
analyze in an handwavy manner the reason for this failure, and give an alternative
algorithm that remains viable in the limit of small discretizations.

\subsection{Notations}
In what follows, we loosely consider the notion of a $\emph{Markov Decision Process}$
$\mathcal{M}$ with discretization step $\deltat$. Such an MDP is the discretization
of a continuous MDP with discretization step $\deltat$. Notably, this means that
contiguous states are at distance $\bigO{\deltat}$ and that instantaneous rewards
have magnitude $\bigO{\deltat}$ for any policy, i.e.
\begin{align}
	\|s_{t + \deltat} - s_t\| &= \bigO{\deltat}\\
	r_t = \reward_t \deltat, &\quad \reward_t = \bigO{1}.
\end{align}
Policies are functions from the state space $\statespace$ to distributions on the
$\actionspace$
\begin{equation}
	\pi\colon \statespace \mapsto \mathcal{D}\left(\actionspace\right).
\end{equation}
The probability of an action $a$ in a given state $s$ under policy $\pi$ is
denoted by $\pi\left(a\mid s\right)$.


%%!TEX root = main.tex
\section{An aside on learning rates}
\label{sec:lr}
Defining $P_{ij} = P(j\mid i)$, $\mathrm{TD}(0)$ can be rewritten as
\begin{align*}
	V^{t+1} - V^* = V^t - V^* + \alpha E_{s_t}\left[{(E_{s_{t+1}} - P^T E_{s_t})}^T R + 
		\gamma E_{s_{t+1}}^T V^t - \gamma E_{s_t}^T P V^* - E_{s_t}^T(V^t-V^*)
	\right]
\end{align*}
or, with $W^t = V^t - V^*$,
\begin{align*}
	W^{t+1} = \left[I + \alpha E_{s_t}{(\gamma E_{s_{t+1}} - E_{s_t})}^T\right] W^t +
	\alpha E_{s_t}{(E_{s_{t+1}} - P^T E_{s_t})}^T (R + \gamma V^*)
\end{align*}
We would like to find conditions on $\alpha$ that turn $\|W^t\|^2$ into a submartingale. 
Rewrite
\begin{align*}
	W^{t+1} = X^t + Y^{t+1}
\end{align*}
where
\begin{align*}
	X^t &= \left[I - \alpha E_{s_t}E_{s_t}^T\right] W^t - 
	\alpha E_{s_t}E_{s_t}^T P{(R + \gamma V^*)}\\
	Y^{t+1} &= \alpha \gamma E_{s_t}E_{s_{t+1}}^T W^t + \alpha E_{s_t}E_{s_{t+1}}^T(R + \gamma V^*)\\
	&= \alpha E_{s_t}E_{s_{t+1}}^T(R + \gamma V^t).
\end{align*}
Recall that $\BE\left[E_{s_{t+1}}\mid \pfield\right] = P^T E_{s_t}$.
\begin{align}
	\BE\left[Y^{t+1}\mid \pfield\right] &= \alpha E_{s_t} E_{s_t}^T P (R + \gamma V^t).
\end{align}
Besides,
\begin{align*}
	\BE\left[E_{s_{t+1}}E_{s_t}^T E_{s_t} E_{s_{t+1}}^T\mid \pfield\right] &=
	\sum\limits_{s_{t+1}} P(s_{t+1}\mid s_t) E_{s_{t+1}} E_{s_{t+1}}^T\\
	&= \mathrm{diag}{(P^T E_{s_t})}
\end{align*}
yielding
\begin{align}
	\BE\left[Y^{t+1}^T Y^{t+1}\mid \pfield\right] &= \alpha ^ 2
	\BE\left[{\left(R(s_{t+1}) + \gamma V^t(s_{t+1})\right)}^2\mid \pfield\right].
\end{align}
Similarily
\begin{align*}
	X^t^T X^t = &\|\left[I - \alpha E_{s_t}E_{s_t}^T\right]W^t\|^2 + 2 \alpha{(\alpha - 1)}
	W^t(s_t) V^*(s_t) + \alpha^2V^*(s_t)^2
\end{align*}
and
\begin{align*}
	\BE\left[X^{tT} Y^{t+1}\mid \pfield\right] = & \alpha(1 - \alpha)
	\BE\left[R(s_{t+1}) + \gamma V^t(s_{t+1})\mid \pfield \right]W^t(s_t)
	- \\&\alpha^2 \BE\left[R(s_{t+1} + \gamma V^t(s_{t+1}))\mid \pfield\right] V^*(s_t).
\end{align*}
Summing, we obtain
\begin{align*}
	\BE\left[\|W^t\|^2\mid \pfield\right] &=
	\|\left[I - \alpha E_{s_t}E_{s_t}^T\right]W^t\|^2
	- \alpha^2 V^*(s_t)^2 \\
	&-2 \alpha (1 - \alpha) \BE\left[\gamma W^t(s_{t+1})\mid \pfield\right] W^t(s_t)
	 \\
	&+2 \alpha^2 \BE\left[\gamma W^t(s_{t+1})\mid \pfield\right] V^*(s_t) +
	\alpha^2
	\BE\left[{\left(R(s_{t+1}) + \gamma V^t(s_{t+1})\right)}^2\mid \pfield\right].
\end{align*}


%!TEX root = main.tex

\section{Continuous MCTS}
\label{sec:mcts}
Pseudo code strongly inspired by \href{http://ccg.doc.gold.ac.uk/ccg_old/teaching/ludic_computing/ludic16.pdf}{this link}.
\begin{algorithm}
	\caption{\textproc{UCTSearch}}
	\begin{algorithmic}[1]
		\Function{UCTSearch}{$s_0$}
		\State create $v_0$ with state $s_0$
		\While{within computational budget}
		\State $v_l \leftarrow \textproc{TreePolicy}(v_0)$
		\State $\Delta \leftarrow \textproc{Eval}(s(v_l))$
		\State $\textproc{Backup}(v_l, \Delta)$
		\EndWhile
		\State \Return $a(\textproc{BestChild}(v_0, 0))$
		\EndFunction
	\end{algorithmic}
\end{algorithm}
\begin{algorithm}
	\caption{\textproc{TreePolicy}}
	\begin{algorithmic}[1]
		\Function{TreePolicy}{$v$}
		\While{$v$ is non terminal}
		\If{$v$ is not expanded or $\textproc{Unif}() < C \deltat$}
		\State \Return $\textproc{Expand}(v)$
		\Else
		\State $v \leftarrow \textproc{BestChild}(v)$
		\EndIf
		\EndWhile
		\State \Return $v$
		\EndFunction
	\end{algorithmic}
\end{algorithm}
\begin{algorithm}
	\caption{\textproc{Expand}}
	\begin{algorithmic}
		\Function{Expand}{$v$}
		\State Create new node $v'$
		\State $a(v') \sim \pi(a\mid s(v))$
		\State $s(v') \sim T(s'\mid s(v), a)$
		\State \Return $v'$
		\EndFunction
	\end{algorithmic}
\end{algorithm}
\begin{algorithm}
	\caption{\textproc{BestChild}}
	\begin{algorithmic}[1]
	\Function{BestChild}{$v$}
	\State \Return $\argmax\limits_{v' \in \textrm{Expanded children of } v}
	\frac{Q(v')}{N(v')} + c \sqrt{\frac{2 \ln{N(v)}}{N(v')}}$
	\EndFunction
	\end{algorithmic}
\end{algorithm}
\begin{algorithm}
	\caption{\textproc{Backup}}
	\begin{algorithmic}
		\Function{Backup}{$v, \Delta$}
		\While{$v$ is not null}
		\State $N(v) \leftarrow N(v) + 1$
		\State $Q(v) \leftarrow Q(v) + \Delta(v)$
		\State $v \leftarrow \mathrm{parent of } v$
                \EndWhile
		\EndFunction
	\end{algorithmic}
\end{algorithm}



%%% Local Variables:
%%% mode: latex
%%% TeX-master: "main"
%%% End:

\section{Conclusion}
\label{sec:conclusion}


\end{document}

%%% Local Variables:
%%% mode: latex
%%% TeX-master: t
%%% End:
